\section{Kết luận}
Báo cáo này đã trình bày chi tiết quá trình triển khai trò chơi Gomoku bằng ngôn ngữ lập trình MIPS assembly. Thông qua việc sử dụng Máy trạng thái hữu hạn (FSM), các thuật toán và cấu trúc dữ liệu được thiết kế một cách rõ ràng và hiệu quả để quản lý luồng trò chơi, xử lý đầu vào, kiểm tra điều kiện thắng hoặc hòa, và ghi kết quả.

Việc triển khai trò chơi đã giúp làm nổi bật các khía cạnh quan trọng của lập trình cấp thấp, bao gồm quản lý bộ nhớ, sử dụng thanh ghi, và tối ưu hóa logic xử lý. Các thủ tục như \texttt{INIT}, \texttt{GET\_MOVE}, \texttt{UPDATE\_BOARD}, và \texttt{check\_win} đã được xây dựng để đảm bảo tính chính xác và hiệu quả trong việc thực thi.

Ngoài ra, báo cáo cũng nhấn mạnh tầm quan trọng của việc xử lý lỗi và đảm bảo trải nghiệm người dùng mượt mà, thông qua các cơ chế kiểm tra đầu vào và thông báo lỗi rõ ràng. Kết quả cuối cùng là một chương trình hoạt động ổn định, đáp ứng đầy đủ các yêu cầu của bài tập lớn.

Qua dự án này, nhóm đã có cơ hội áp dụng kiến thức về Kiến trúc máy tính vào một ứng dụng thực tế, đồng thời nâng cao kỹ năng lập trình và tư duy giải quyết vấn đề. Đây là một trải nghiệm quý báu, giúp nhóm hiểu sâu hơn về cách các hệ thống máy tính hoạt động ở mức thấp và cách tối ưu hóa chúng để đạt hiệu suất cao.