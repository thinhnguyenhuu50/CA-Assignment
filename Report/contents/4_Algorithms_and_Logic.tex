\section{Các giải thuật và logic}
\subsection{Nhận tọa độ từ người chơi}
\begin{enumerate}
    \item \textbf{Nhận tọa độ di chuyển của người chơi trong trò chơi Gomoku}: Chương trình thực hiện một quy trình có cấu trúc để đảm bảo rằng đầu vào được định dạng đúng, nằm trong phạm vi hợp lệ và trùng với một vị trí trên bàn cờ chưa bị chiếm.
    \item \textbf{Hiển thị thông báo yêu cầu người chơi}: 
    \begin{itemize}
        \item Đối với Người chơi 1: "Người chơi 1, vui lòng nhập tọa độ của bạn: "
        \item Đối với Người chơi 2: "Người chơi 2, vui lòng nhập tọa độ của bạn: "
    \end{itemize}
\end{enumerate}

\begin{code}
    # Display prompt based on current player
    li $v0, 4         # Syscall to print string
    beq $s0, 1, print_prompt1
    la $a0, prompt2   # "Player 2, please input your coordinates: "
    j print_prompt
print_prompt1:
    la $a0, prompt1   # "Player 1, please input your coordinates: "
print_prompt:
    syscall
\end{code}

\begin{enumerate}[resume]    
    \item \textbf{Đọc đầu vào}: 
    \begin{itemize}
        \item Chương trình nhận đầu vào từ người dùng dưới dạng chuỗi thông qua một lệnh gọi hệ thống.
        \item Dữ liệu nhận được được lưu vào một bộ đệm để xử lý tiếp.
    \end{itemize}
\end{enumerate}

\begin{code}
    # Read input string
    li $v0, 8         # Syscall to read string
    la $a0, input_buffer
    li $a1, 10        # Max length of input
    syscall
\end{code}

\begin{enumerate}[resume]
    \item \textbf{Phân tích đầu vào}: 
    \begin{itemize}
        \item Phân tách chuỗi để trích xuất tọa độ x và y theo định dạng ``x,y'', trong đó x và y là số nguyên từ 0 đến 14.
        \item \textbf{Trích xuất x}: Đọc các ký tự trước dấu phẩy và kiểm tra rằng chúng đều là chữ số, xây dựng giá trị x bằng cách nhân giá trị hiện tại với 10 và cộng thêm chữ số mới.
        \item \textbf{Bỏ qua dấu phẩy}: Di chuyển qua dấu phẩy để xử lý tiếp phần tọa độ y.
        \item \textbf{Trích xuất y}: Đọc các chữ số sau dấu phẩy cho đến khi gặp ký tự xuống dòng hoặc ký tự kết thúc, xây dựng giá trị y tương tự như x.
        \item Nếu bất kỳ ký tự nào không phải là số hoặc định dạng không đúng (ví dụ, thiếu dấu phẩy), đầu vào sẽ bị xem là không hợp lệ.
    \end{itemize}
\end{enumerate}

\begin{code}
parse_x:
    lb $t0, 0($s0)    # Load current character
    beq $t0, 44, end_parse_x  # If comma (ASCII 44), end x parsing
    beq $t0, 0, invalid  # If null, invalid
    beq $t0, 10, invalid # If newline, invalid

    # Check if character is a digit (0-9)
    blt $t0, 48, invalid  # < '0'
    bgt $t0, 57, invalid  # > '9'

    # Update x = x * 10 + (char - '0')
    mul $s1, $s1, 10  # x *= 10
    sub $t0, $t0, 48  # Convert char to integer
    add $s1, $s1, $t0 # x += digit

    addi $s0, $s0, 1  # Move to next character
    j parse_x

end_parse_x:
    addi $s0, $s0, 1  # Skip comma
    # Step 2: Parse y (digits after comma)
parse_y:
    lb $t0, 0($s0)    # Load current character
    beq $t0, 10, end_parse_y  # If newline, end y parsing
    beq $t0, 0, end_parse_y   # If null, end y parsing

    # Check if character is a digit
    blt $t0, 48, invalid  # < '0'
    bgt $t0, 57, invalid  # > '9'

    # Update y = y * 10 + (char - '0')
    mul $s2, $s2, 10  # y *= 10
    sub $t0, $t0, 48  # Convert char to integer
    add $s2, $s2, $t0 # y += digit

    addi $s0, $s0, 1  # Move to next character
    j parse_y

end_parse_y:
\end{code}

\begin{enumerate}[resume]
    \item \textbf{Kiểm tra tính hợp lệ của tọa độ}:
    \begin{itemize}
        \item \textbf{Kiểm tra phạm vi}: Xác nhận rằng cả x và y đều nằm trong phạm vi từ 0 đến 14, tương ứng với bàn cờ 15x15.
        \item \textbf{Kiểm tra vị trí}: Đảm bảo rằng vị trí tại tọa độ (x, y) trên bàn cờ đang trống.
    \end{itemize}
\end{enumerate}

\begin{code}
    blt $s1, 0, invalid   # x < 0
    bgt $s1, 14, invalid  # x > 14
    blt $s2, 0, invalid   # y < 0
    bgt $s2, 14, invalid  # y > 14

    # Step 4: Check if board position is empty
    get($s1, $s2, $v0)
    bne $v0, empty, invalid  # If not empty, position is occupied

    # Step 5: Valid input, return x, y
    move $v0, $s1         # Return x
    move $v1, $s2         # Return y
    j parse_input_exit
\end{code}

\begin{enumerate}[resume]
    \item \textbf{Xử lý đầu vào không hợp lệ}: Nếu đầu vào không hợp lệ do định dạng sai như ``a,b'' hoặc `4', giá trị vượt phạm vi như `15,2', hoặc vị trí đã bị chiếm, chương trình sẽ hiển thị thông báo lỗi `Đầu vào không hợp lệ, vui lòng nhập lại' và quay lại bước 1 để yêu cầu người chơi nhập lại.
\end{enumerate}

\begin{code}
    # Check if input is invalid
    li $t0, -1
    beq $v0, $t0, invalid_input

    # Input is valid, return x,y
	# x already in v0
	# y in v1
    j get_move_exit

invalid_input:
    # Print error message
    li $v0, 4
    la $a0, error_msg # "Invalid input, try again"
    syscall
    j input_loop      # Re-prompt
\end{code}

\begin{enumerate}[resume]
    \item \textbf{Trả về tọa độ hợp lệ}: Khi đầu vào vượt qua tất cả các kiểm tra, chương trình trả về tọa độ (x, y) để sử dụng trong logic trò chơi, ghi nhận nước đi của người chơi trên bàn cờ.
\end{enumerate}

Quy trình trên đảm bảo rằng chỉ những nước đi hợp lệ được chấp nhận, đáp ứng các yêu cầu của bàn cờ 15x15 và định dạng đầu vào dự kiến, mang lại trải nghiệm người dùng thân thiện và chính xác.