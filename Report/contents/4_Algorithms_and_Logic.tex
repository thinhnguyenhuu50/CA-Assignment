\section{Các giải thuật và logic}
\subsection{Nhận tọa độ từ người chơi}
\begin{enumerate}
    \item \textbf{Nhận tọa độ di chuyển của người chơi trong trò chơi Gomoku}: Chương trình thực hiện một quy trình có cấu trúc để đảm bảo rằng đầu vào được định dạng đúng, nằm trong phạm vi hợp lệ và trùng với một vị trí trên bàn cờ chưa bị chiếm.
    \item \textbf{Hiển thị thông báo yêu cầu người chơi}: 
    \begin{itemize}
        \item Đối với Người chơi 1: ``Người chơi 1, vui lòng nhập tọa độ của bạn: ''
        \item Đối với Người chơi 2: ``Người chơi 2, vui lòng nhập tọa độ của bạn: ''
    \end{itemize}
\end{enumerate}

\begin{code}
    # Display prompt based on current player
    li $v0, 4         # Syscall to print string
    beq $s0, 1, print_prompt1
    la $a0, prompt2   # "Player 2, please input your coordinates: "
    j print_prompt
print_prompt1:
    la $a0, prompt1   # "Player 1, please input your coordinates: "
print_prompt:
    syscall
\end{code}

\begin{enumerate}[resume]    
    \item \textbf{Đọc đầu vào}: 
    \begin{itemize}
        \item Chương trình nhận đầu vào từ người dùng dưới dạng chuỗi thông qua một lệnh gọi hệ thống.
        \item Dữ liệu nhận được được lưu vào một bộ đệm để xử lý tiếp.
    \end{itemize}
\end{enumerate}

\begin{code}
    # Read input string
    li $v0, 8         # Syscall to read string
    la $a0, input_buffer
    li $a1, 10        # Max length of input
    syscall
\end{code}

\begin{enumerate}[resume]
    \item \textbf{Phân tích đầu vào}: 
    \begin{itemize}
        \item Phân tách chuỗi để trích xuất tọa độ x và y theo định dạng ``x,y'', trong đó x và y là số nguyên từ 0 đến 14.
        \item \textbf{Trích xuất x}: Đọc các ký tự trước dấu phẩy và kiểm tra rằng chúng đều là chữ số, xây dựng giá trị x bằng cách nhân giá trị hiện tại với 10 và cộng thêm chữ số mới.
        \item \textbf{Bỏ qua dấu phẩy}: Di chuyển qua dấu phẩy để xử lý tiếp phần tọa độ y.
        \item \textbf{Trích xuất y}: Đọc các chữ số sau dấu phẩy cho đến khi gặp ký tự xuống dòng hoặc ký tự kết thúc, xây dựng giá trị y tương tự như x.
        \item Nếu bất kỳ ký tự nào không phải là số hoặc định dạng không đúng (ví dụ, thiếu dấu phẩy), đầu vào sẽ bị xem là không hợp lệ.
    \end{itemize}
\end{enumerate}

\begin{code}
parse_x:
    lb $t0, 0($s0)    # Load current character
    beq $t0, 44, end_parse_x  # If comma (ASCII 44), end x parsing
    beq $t0, 0, invalid  # If null, invalid
    beq $t0, 10, invalid # If newline, invalid

    # Check if character is a digit (0-9)
    blt $t0, 48, invalid  # < '0'
    bgt $t0, 57, invalid  # > '9'

    # Update x = x * 10 + (char - '0')
    mul $s1, $s1, 10  # x *= 10
    sub $t0, $t0, 48  # Convert char to integer
    add $s1, $s1, $t0 # x += digit

    addi $s0, $s0, 1  # Move to next character
    j parse_x

end_parse_x:
    addi $s0, $s0, 1  # Skip comma
    # Step 2: Parse y (digits after comma)
parse_y:
    lb $t0, 0($s0)    # Load current character
    beq $t0, 10, end_parse_y  # If newline, end y parsing
    beq $t0, 0, end_parse_y   # If null, end y parsing

    # Check if character is a digit
    blt $t0, 48, invalid  # < '0'
    bgt $t0, 57, invalid  # > '9'

    # Update y = y * 10 + (char - '0')
    mul $s2, $s2, 10  # y *= 10
    sub $t0, $t0, 48  # Convert char to integer
    add $s2, $s2, $t0 # y += digit

    addi $s0, $s0, 1  # Move to next character
    j parse_y

end_parse_y:
\end{code}

\begin{enumerate}[resume]
    \item \textbf{Kiểm tra tính hợp lệ của tọa độ}:
    \begin{itemize}
        \item \textbf{Kiểm tra phạm vi}: Xác nhận rằng cả x và y đều nằm trong phạm vi từ 0 đến 14, tương ứng với bàn cờ 15x15.
        \item \textbf{Kiểm tra vị trí}: Đảm bảo rằng vị trí tại tọa độ (x, y) trên bàn cờ đang trống.
    \end{itemize}
\end{enumerate}

\begin{code}
    blt $s1, 0, invalid   # x < 0
    bgt $s1, 14, invalid  # x > 14
    blt $s2, 0, invalid   # y < 0
    bgt $s2, 14, invalid  # y > 14

    # Step 4: Check if board position is empty
    get($s1, $s2, $v0)
    bne $v0, empty, invalid  # If not empty, position is occupied

    # Step 5: Valid input, return x, y
    move $v0, $s1         # Return x
    move $v1, $s2         # Return y
    j parse_input_exit
\end{code}

\begin{enumerate}[resume]
    \item \textbf{Xử lý đầu vào không hợp lệ}: Nếu đầu vào không hợp lệ do định dạng sai như ``a,b'' hoặc ``4'', giá trị vượt phạm vi như ``15,2'', hoặc vị trí đã bị chiếm, chương trình sẽ hiển thị thông báo lỗi ``\texttt{invalid input, try again}'' và quay lại bước 1 để yêu cầu người chơi nhập lại.
\end{enumerate}

\begin{code}
    # Check if input is invalid
    li $t0, -1
    beq $v0, $t0, invalid_input
    # Input is valid, return x,y
	# x already in $v0
	# y in $v1
    j get_move_exit

invalid_input:
    # Print error message
    li $v0, 4
    la $a0, error_msg # "Invalid input, try again"
    syscall
    j input_loop      # Re-prompt
\end{code}



\begin{enumerate}[resume]
    \item \textbf{Trả về tọa độ hợp lệ}: Khi đầu vào vượt qua tất cả các kiểm tra, chương trình trả về tọa độ (x, y) để sử dụng trong logic trò chơi, ghi nhận nước đi của người chơi trên bàn cờ.
\end{enumerate}

Quy trình trên đảm bảo rằng chỉ những nước đi hợp lệ được chấp nhận, đáp ứng các yêu cầu của bàn cờ 15x15 và định dạng đầu vào dự kiến, mang lại trải nghiệm người dùng thân thiện và chính xác.

\subsection{Kiểm tra kết quả trận đấu}
Điều kiện để người chơi dành chiến thắng trong trò chơi Gokumo:
\begin{itemize}
    \item \textbf{Điều kiện 1:} Người chơi thắng khi có 5 quân liên tiếp cùng loại (theo hàng ngang, dọc hoặc chéo) trên bàn cờ 15x15.
    \item \textbf{Điều kiện 2:} Người chơi phải hoàn thành chuỗi 5 quân liên tiếp này trong một khoảng thời gian giới hạn theo quy định của trò chơi. Nếu hết thời gian quy định thì người chơi sẽ bị xử thua.
    \item \textbf{Điều kiện 3:} Nếu trận đấu tàn cuộc hay bàn cờ đầy dẫn đến kết quả hòa thì nếu thời gian của người chơi nào còn nhiều hơn thì sẽ được xử thắng.
    \item \textbf{Điều kiện 4:} Ngoài ra, nếu người chơi đầu hàng thì kết quả chiến thắng sẽ thuộc về người còn lại.  
\end{itemize}

\textbf{Lưu ý:} 
\begin{itemize}
\item Nếu trong setting của trò chơi không có quy định về luật giới hạn thời gian thì xem như bỏ qua điều kiện 2 và 3.

\item Thứ tự kiểm tra sẽ ưu tiên điều kiện \(4 \rightarrow 2 \rightarrow 1 \rightarrow 3\). Điều kiện nào được ưu tiên hơn sẽ được thực hiện trước và nếu xảy ra sẽ dẫn đến kết quả của điều kiện đó.
\end{itemize}

Trò chơi sẽ được xử hòa nếu số lượt đi của trận đấu là 225 hay bàn cờ đầy. Nếu có luật thời gian thì thời gian 2 người phải bằng nhau.

\subsubsection{Kiểm tra điều kiện thứ nhất}
Hàm \var{check\_win} sẽ kiểm tra điều kiện đầu của trò chơi như sau:
\begin{enumerate}
\item \textbf{Tham số đầu vào :} Hàm nhận 2 tham số \var{\$a0}, \var{\$a1} lần lượt là tọa độ x và y của lượt vừa đi của người chơi để kiểm tra điều kiện chiến thắng.  
\item \textbf{Lưu thanh ghi vào stack trước khi xử lí:} Lưu trữ dữ liệu của các thanh ghi tham số \var{\$a0}, \var{\$a1}, thanh ghi \var{\$ra} và \var{\$s0}, \(\dots\) để sử dụng nhằm tránh mất mát dữ liệu khi gọi hàm.
\item \textbf{Lấy giá trị tại tọa độ của bàn cờ:} gọi macro \var{get} để lấy giá trị của bàn cờ tại tọa độ x và y.
\item \textbf{Kiểm tra điều kiện thắng của trò chơi:} 
    \begin{itemize}
        \item Từ tọa độ của người ta phải kiểm tra 4 hướng chính như sau: \(\searrow \) \(\nearrow\) \(\uparrow\) \(\to \). Hàm sẽ sử dụng vòng lặp để chạy 4 hướng chính sau cũng như tạo sẵn 2 mảng dùng để dễ dàng tăng thêm theo hướng cho tọa độ x, y đang đứng. 
        \item Dùng vòng lặp để chạy cả 2 chiều theo chiều dương và chiều âm của hướng để kiểm tra lần lượt các điều kiện sau để dừng vòng lặp:
        \begin{itemize}
            \item Ô đang đứng có phải là vị trí hợp lệ hay không, tức là kiểm tra \(0 \leq x \leq 14\) và \(0 \leq y \leq 14\)
            \item So sánh giá trị của bàn cờ tại tọa độ đang đứng có bằng với giá trị của người chơi hay không. \textit{Ví dụ}: Hàm có đầu vào ban đầu là \((x_0, y_0)\) có giá trị trên bàn cờ là \textbf{X}, sau khi chạy vòng lặp tới ô (\(x_1, y_1\)) nếu có giá trị bàn cờ là \textbf{X} thì tiếp tục còn là \textbf{O} thì dừng.
        \end{itemize}
        \item Sau khi kiểm tra điều kiện thỏa mãn, ta tăng tổng ô liên tiếp 1 đơn vị và kiểm tra thêm nếu tổng lớn hơn bằng 5 thì ngừng lặp và trả về kết quả true. Nếu không thì tiếp tục vòng lặp. Khi hết mỗi hướng đi thì ta reset lại tổng đó về 1 để đếm lại.
        \item Sau khi hết vòng lặp mà chưa trả về kết quả thì hàm trả về fasle. 
    \end{itemize}
    \item \textbf{Load dữ liệu của thanh ghi trong stack sau khi xử lí:} Load dữ liệu của các thanh ghi \var{\$ra} và \var{\$s0}, \(\dots\) nhằm tránh mất mát dữ liệu khi gọi hàm.
\item \textbf{Kết quả trả về} \var{\$v0}: 
\begin{itemize}
\item Nếu là 1 (true) thì điều kiện này dẫn đến chiến thắng cho người chơi. 
\item Các kết quả thì xem như bỏ qua và kiểm tra tiếp điều kiện tiếp theo.
\end{itemize}
\end{enumerate}


\subsubsection{Kiểm tra điều kiện thứ 2}
Hàm \var{check\_time} sẽ kiểm tra điều kiện thời gian của trò chơi như sau:
\begin{enumerate}
\item \textbf{Tham số đầu vào:} 
\begin{itemize}
\item \var{\$a0}: Là người chơi vừa đi lượt trên. Nếu là 1 thì đó là người thứ 1 còn nếu là 2 thì đó là người chơi thứ 2.
\item \var{\$a1}: Là địa chỉ đầu của mảng integer \var{timer} biểu diễn dữ liệu thời gian của trò chơi.
\begin{itemize}
\item \var{timer[0]}: Là luật chơi của trận đấu có bị giới hạn bởi thời gian hay không. Nếu là 1 là có còn 0 là không.
\item \var{timer[1]}: Là thời gian còn lại của người chơi thứ 1 (tính bằng giây).
\item \var{timer[2]}: Là thời gian còn lại của người chơi thứ 2 (tính bằng giây).
\end{itemize}
\end{itemize}
\item \textbf{Kiểm tra xem trận đấu có luật giới hạn thời gian:} Nếu không có giới hạn thì trả về kết quả là 1.
\item \textbf{Kiểm tra thời gian còn lại của người chơi vừa đi:} Nếu thời gian lớn hơn 0 thì trả về 1 ngược lại trả về 0
\item \textbf{Kết quả trả về} \var{\$v1}
\begin{itemize}
\item Nếu là 1 thì coi như bỏ qua điều kiện này.
\item Nếu là 0 thì người chơi vừa đi sẽ bị xử thua.
\end{itemize}   
\end{enumerate}
\subsubsection{Kiểm tra điều kiện thứ 3}
Hàm \var{check\_tie} này được thực hiện trong \var{CHECK\_TIE} để kiểm tra điều kiện như sau:
\begin{enumerate}
   \item \textbf{Tham số đầu vào:}
   \begin{itemize}
      \item \var{\$a0} là số lượt đi của trận đấu \var{move\_count}
      \item \var{\$a1} là địa chỉ đầu mảng \var{timer} như trên
   \end{itemize}
   \item \textbf{Nếu trận đấu không có luật giới hạn thời gian:} Nếu số lượt đi \var{\$a0} nhỏ hơn 225 bước thì trả về 3 ngược lại trả về 0.
   \item \textbf{Nếu trận đấu có luật giới hạn thời gian:} Nếu số lượt đi nhỏ hơn 225 thì cũng trả về 3 ngược lại
   \begin{itemize}
      \item Nếu thời gian của hai người chơi bằng nhau thì trả về 0
      \item Nếu thời gian của người chơi nào nhiều hơn thì trả về 1 hoặc 2 tương ứng. 
   \end{itemize}
   \item \textbf{Kết quả trả về} \var{\$v0}:
   \begin{itemize}
      \item Nếu là 0 thì sẽ được xử hòa.
      \item Nếu là 1 thì người chơi 1 thắng.
      \item Nếu là 2 thì người chơi 2 thắng.
      \item Các kết quả còn lại được xem như bỏ qua điều kiện này.
   \end{itemize}
   
\end{enumerate}

\subsubsection{Điều kiện thứ 4}
Nếu trong bước nhập bước đi người chọn lựa đầu hàng trong thanh điều hướng \var{pause} thì người chơi sẽ bị xử thua.

\subsection{Thanh điều hướng}
Trò chơi Gokumo được thiết kế chủ yếu có 3 thanh điều hướng: \var{MENU}, \var{SETTING}, \var{pause} 
\subsubsection{Trang chủ}
Thanh điều hướng \var{MENU} được hiện thực dựa trên hàm \var{menu} như sau:
\begin{enumerate}
   \item \textbf{In ra giao diện Menu}
   \item \textbf{Nhận đầu vào từ bàn phím}
   \item \textbf{Điều hướng}
   \begin{itemize}
      \item \textbf{Nhập 0}: Dừng chương trình.
      \item \textbf{Nhập 1}: Nhảy đến nhãn tạo trận mới \textit{new:}
      \item \textbf{Nhập 2}: Nhảy đến nhãn tiếp tục \textit{load:}
      \item \textbf{Nhập 3}: Đi đến thanh điều hướng \var{SETTING}
      \item \textbf{Các kết quả khác}: In ra và yêu cầu người chơi nhập lại.
   \end{itemize}
\end{enumerate}
\subsubsection{Cài đặt}
Thanh điều hướng \var{SETTING} được hiện thực dựa trên hàm \var{setting} như sau:
\begin{enumerate}
   \item \textbf{In ra giao diện Setting}
   \item \textbf{Nhận đầu vào từ bàn phím}
   \item \textbf{Điều hướng}
   \begin{itemize}
      \item \textbf{Nhập 0}: Quay trở lại menu.
      \item \textbf{Nhập 1}: Nhảy đến nhãn cài đặt luật chơi có giới hạn thời gian bằng cách yêu cầu nhập y/n để chỉnh mảng và ghi vào file save/setting.txt.
      \item \textbf{Nhập 2}: Nhảy đến nhãn điều chỉnh thời gian ban đầu của người chơi 1 và yêu cầu nhập thời gian để chỉnh mảng và ghi vào file save/setting.txt.
      \item \textbf{Nhập 3}: Tương tự như 2 nhưng mà chỉnh mảng cho người chơi 2.
      \item \textbf{Nhập 4}: Reset mảng về trạng thái đầu là có tính luật thời gian và mỗi người có 30 phút rồi ghi vào file.
      \item \textbf{Các kết quả khác}: In ra và yêu cầu người chơi nhập lại.
   \end{itemize}
\end{enumerate}

\subsubsection{Tạm dừng trận đấu}
Thanh điều hướng này được gọi trong thao tác nhập tọa độ người chơi \var{get\_move} mà nhập kí tự 'P'. Thanh điều hướng được hiện thực dựa trên hàm \var{pause\_game} như sau:
\begin{enumerate}
   \item \textbf{Đầu vào}: 
   \begin{itemize}
   \item \var{\$a0}: Người chơi đang trong lượt đánh.
   \item \var{\$a1}: Địa chỉ đầu của mảng timer của trận đấu.
   \item \var{\$a2}: Địa chỉ đầu của biến chứa mốc thời gian trước khi gọi hàm \var{get\_move}.
   \end{itemize}
   \item \textbf{Cập nhật thời gian}: Trừ đi thời gian của người từ khi đổi lượt đến nay và cập nhật mốc thời gian đó bằng thời gian sau khi hàm này trở về trận đấu.  
   \item \textbf{In ra giao diện Menu}
   \item \textbf{Nhận đầu vào từ bàn phím}
   \item \textbf{Điều hướng}
   \begin{itemize}
      \item \textbf{Nhập 0}: Dừng chương trình.
      \item \textbf{Nhập 1}: Trở về trận đấu hay jr \var{\$ra} về lại hàm \var{get\_move}.
      \item \textbf{Nhập 2}: Nhảy đến nhãn đầu hàng \textit{surrender:} để nhận thua.
      \item \textbf{Nhập 3}: Nhảy đến nhãn để thực hiện hàm \var{undo} rồi nhảy về nhãn \textit{load:} để trở về nước trước đó.
      \item \textbf{Nhâp 4}: Trở về thanh điều hướng \var{MENU}.
      \item \textbf{Các kết quả khác}: In ra và yêu cầu người chơi nhập lại.
   \end{itemize}
\end{enumerate}

\subsection{Thao tác tính toán thời gian}
\subsubsection{Bố cục của biến thời gian trong trò chơi}

Để bố cục lại thời gian của một trận đấu ta dùng 2 mảng:
\begin{itemize}
   \item \textit{timer}: chứa thời gian của 2 người chơi và có luật thời gian hay không như ở trước đã trình bày.
   \item \textit{cur\_time}: dùng để lưu mốc thời gian.
\end{itemize}

\subsubsection{Phương thức lấy thời gian thực}
Để lấy thời gian thực ta sử dụng macro \var{get\_time} được trình bày sau:
\begin{code}
.macro get_time(%x) # x = time
   li $v0, 30
	syscall
	move $t0, $a0 
	li $t1, 1000
	div $t0, $t1
	mflo $t0
	li $t1, 32000
	div $t0, $t1
	mfhi %x
.end_macro 
\end{code}
Theo đó thời gian bị giới hạn 32000 giây để tránh trường hợp xử lí bị tràn số gây sai sót kết quả.

\subsubsection{Cách thức đếm thời gian}
\begin{enumerate}
   \item Khi bắt đầu mỗi trận đấu mới chương trình luôn chạy hàm \var{INIT\_SETTING} để lấy những cài đặt trong file \var{setting.txt} về timer.
   \item Nếu là tiếp tục trò chơi thì sẽ lấy timer từ file \var{playertime.txt} qua phương thức \var{LOAD}.
   \item Trong macro \var{GET\_MOVE}, ta thực hiện lưu mốc thời gian vào biến \var{cur\_time} trước rồi gọi hàm \var{get\_move}. Trong thời gian thực hiện mà người chơi muốn dừng trận đấu lại thì thời gian sẽ được giữ nguyên bằng cách trừ thời gian còn lại của người chơi với hiệu của mốc thời gian hiện tại và mốc \var{cur\_time}. Sau khi pause trở về thì cập nhật lại mốc thời gian \var{cur\_time} thành hiện tại. Sau khi gọi hàm \var{get\_move} thì ta cũng thực hiện trừ tương tự cho thời gian của người chơi
\end{enumerate}

\subsection{Thao tác lưu trữ và đọc tập tin}
Trò chơi Gokumo được bổ sung thêm tính năng tải lại ván cờ vừa chơi (load game), đi lại bước vừa đi (undo) và các cài đặt của setting nên việc lưu trữ các thông tin của ván cờ vào file là không thể thiếu.

\subsubsection{Cấu trúc dữ liệu của file save}
Tập tin lưu trữ của trò chơi bao gồm:
\begin{enumerate}
   \item \var{setting.txt}: File gồm 1 mảng integer có kích thước là 12 byte định dạng giống như mảng \var{timer} ở trên. Để lưu trữ các cài đặt của setting và làm giá trị đầu cho \var{timer} mỗi trận.
   \item \var{playertime.txt}: File có định dạng giống y chang file \var{setting.txt}. Mục đích của file này là để lưu lại biến timer của trận trước và dùng để load lại \var{timer} khi tải lại trận trước.
   \item \var{board.txt}: File này có định dạng là 1 mảng integer được cấp phát động. Mỗi phần tử thứ i là tọa độ của người chơi tại lượt đánh thứ i+1 trên board. Ví dụ như ở lượt đánh thứ 5 có tọa độ (3,4) thì \var{board[4]} = 50 + 3*47 + 4*3 = 203. Dễ thấy kích thước của file là bằng tổng số lượt nhân 4 byte. Mỗi ô thứ i chẵn là nước đi của player 1 ngược lại là player 2. Mỗi tọa độ của từng lượt đi được sắp xếp đúng cách khiến việc tải lại thông tin của trận đấu trở nên dễ dàng và hợp lí.  
\end{enumerate}

\subsubsection{Thao tác lưu trữ}
Thao tác lưu trữ trận đấu mỗi nước đi được thực hiện qua hàm \var{save\_game} như sau:
\begin{enumerate}
   \item \textbf{Tham số đầu vào}: 
   \begin{itemize}
      \item \var{\$a0}: Tọa độ x của nước đi vừa rồi của người chơi.
      \item \var{\$a1}: Tọa độ y của nước đi vừa rồi của người chơi.
      \item \var{\$a2}: Địa chỉ đầu của biến \var{timer} của lượt đấu.
   \end{itemize}
   \item \textbf{Lưu trữ và tải lại thanh ghi sử dụng vào stack}
   \item \textbf{Tính toán vị trí chính xác trên board}: Giá trị dược tính = \(50 + x*47 + y*3\)
   \item \textbf{Lưu tiếp (append) vào file }\var{board.txt}
   \item \textbf{Lưu mảng }\var{timer} \textbf{vào file} \var{playertime.txt}
\end{enumerate}

\subsubsection{Thao tác tải lại}
Thao tác tải lại trận đấu cũ được thực hiện qua hàm \var{load\_game} như sau:
\begin{enumerate}
   \item \textbf{Tham số đầu vào} \var{\$a0} Địa đầu của mảng timer để load lại.
   \item \textbf{Lưu trữ và tải lại thanh ghi sử dụng vào stack}
   \item \textbf{Tải lại bàn cờ}: 
   \begin{itemize}
      \item Trước khi tải phải gọi macro \var{INIT} và \var{INIT\_SETTING} để cấp phát giá trị ban đầu cho bàn cờ và \var{timer}. 
      \item Đọc file \var{board.txt} vào buffer. Lặp buffer nếu i chẵn thì ghi trực tiếp X vào board qua sb ngược lại ghi O. Lưu lại kích thước đã đọc để chia 4 để lưu vào \var{\$v1}.
      \item Đọc file \var{setting.txt} vào buffer. Kiểm tra xem nếu không tính luật thời gian thì không cần phải đọc timer.
      \item Đọc file \var{playertime.txt} vào địa chỉ của mảng \var{timer}.
   \end{itemize}
   \item \textbf{Kết quả trả về}:
   \begin{itemize}
   \item \var{\$v0}: player sẽ được đi tiếp theo. Nếu tổng số lượt là chẵn thì player 1 hay trả về 1 ngược lại trả về 2.
   \item \var{\$v1}: Tổng số lượt đi của trò chơi để gán \var{move\_count}
   \end{itemize}
\end{enumerate}
\subsubsection{Thao tác đi lại}
Thao tác đi lại được thực hiện qua hàm \var{undo} như sau:
\begin{enumerate}
   \item \textbf{Lưu trữ và tải lại thanh ghi sử dụng vào stack}
   \item Đọc file \var{board.txt} vào buffer và lưu trữ kích thước đã đọc
   \item Nếu kích thước bằng 0 thì trả về nếu không thì trừ đi 4 và vào file \var{board.txt}
   \item Sau khi hàm trả về thì sẽ nhảy sang nhãn \textit{load} để tải lại trận đấu. Các tham số thời gian sẽ bị giữ nguyên chứ không bị ghi lại từ lượt trước.
\end{enumerate}
\subsubsection{Thao tác xóa file}
Sau khi mỗi lần trận đấu bị dừng do có kết quả thắng hoặc hòa hay bắt đầu trận mới thì sẽ gọi hàm để xóa file qua macro \var{CLEAR\_FILE\_LOAD} gọi các hàm như \var{clear\_load}, \var{init\_setting} và \var{clear\_time}
\begin{enumerate}
   \item Hàm \var{clear\_load} xóa file bằng cách ghi đè mảng rỗng vào file \var{board.txt}
   \item Hàm \var{init\_setting} khiến cho mảng timer về lại trạng đầu của file \var{setting.txt}
   \item Hàm \var{clear\_time} ghi đè mảng \var{timer} vừa bị ghi đè vào file \var{playertime}
\end{enumerate}
\subsection{Ghi kết quả trò chơi}
Sau khi đã xác định kết quá của trò chơi (hòa hoặc có người chơi thắng), chương trình thực hiện ghi kết quả ra console và file ``result.txt''. 

\subsubsection*{Ghi kết quả chiến thắng: } Chương trình thực hiện tuần tự các bước sau:
\begin{enumerate}
    \item \textbf{Lưu thanh ghi vào stack:} Để đảm bảo toàn vẹn dữ liệu giống như đã nêu ở phần kiểm tra chiến thắng, chương trình thực hiện lưu lại các thanh ghi như sau:
    \begin{code}
        win_process:
            # Save registers
            addi $sp, $sp, -12
            sw $ra, 0($sp)
            sw $s0, 4($sp)      # for file descriptor
            sw $t0, 8($sp)      # for msg addr
        \end{code}
        Do ta cần dùng thanh ghi \var{\$s0} cho mô tả tập tin, và thêm 1 thanh ghi \var{\$t0} để lưu địa chỉ của thông báo nên chương trình thực hiện lưu thanh ghi \var{\$ra} cùng với 2 thanh ghi này.
    \item \textbf{Xác định người chơi nào thắng cuộc:} 
    \begin{code}
        # Determine winner msg
        bne $s7, 1, p2_wins # $s7 = player
        la $a0, p1_win_msg
        la $t0, p1_win_msg  # store msg addr for later
        j write_result
    p2_wins:
        la $a0, p2_win_msg
        la $t0, p2_win_msg  # store msg addr for later
    \end{code}

    \item \textbf{Ghi kết quả ra console:}
    \begin{code}
write_result:
    # Print winner msg to console
    li $v0, 4
    syscall
    \end{code}
    \item \textbf{Ghi kết quả vào file ``result.txt'':} Ta cần thực hiện các bước: mở file, ghi thông báo người thắng, ghi bàn cờ, đóng file. Các bước trên được thực hiện bằng đoạn chương trình sau:
    \begin{code}
    # Open file "result.txt"
    li $v0, 13
    la $a0, result_file
    li $a1, 9           # Write-only, create, append
    li $a2, 438         # File mode
    syscall
    move $s0, $v0       # Save file descriptor      

    # Write winner msg to file
    li $v0, 15
    move $a0, $s0
    move $a1, $t0       # load msg addr from $t0
    li $a2, 14          # Size of "Player X wins\n"
    syscall

    # Write board to file
    li $v0, 15
    move $a0, $s0
    la $a1, board
    li $a2, 752         # Size of the board
    syscall

    # Close file
    li $v0, 16
    move $a0, $s0
    syscall

    \end{code}
    Do giá trị thanh ghi \var{\$a0} và \var{\$a1} thay đổi liên tục xuyên suốt các bước, nên các thanh ghi \var{\$s0} cũng như \var{\$t0} là cần thiết để chương trình thực thi chính xác.
    
    
    Ngoài ra, khi ghi thông báo vào file, vì cả 2 thông báo ``Player 1 wins\textbackslash n''   và ``Player 2 wins\textbackslash n'' có cùng độ dài 14 ký tự, nên ta chỉ cần load integer 14 vào thanh ghi \(\$a2\) cho cả 2 trường hợp. 
    \item \textbf{Trả dữ liệu của các thanh ghi về như cũ:}
    \begin{code}
    # Restore registers
    lw $t0, 8($sp)       
    lw $s0, 4($sp)
    lw $ra, 0($sp)
    addi $sp, $sp, 12
    jr $ra
    \end{code}
    Sau khi thực hiện xong việc ghi kết quả chiến thắng, chương trình trả dữ liệu của các thanh ghi \var{\$ra}, \var{\$s0}, \var{\$t0}, tiếp tục thực thi những lệnh kế tiếp.
\end{enumerate}
\subsubsection*{Ghi kết quả hòa: } Chương trình thực hiện các bước tương tự như ghi kết quả chiến thắng.